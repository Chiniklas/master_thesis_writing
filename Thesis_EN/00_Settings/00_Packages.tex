\documentclass[%
	a4paper, % Papierformat
	12pt, % Standard-Schriftgröße
	DIV=14, % Seitenspiegel angepasst (bei DIV=calc wird DIV nach der Schriftdefinition berechnet mit \typearea[current]{calc}
	twoside, % gespiegelte Seitenränder
%	open=right, % Kapitel immer auf der rechten Seite beginnen
	BCOR=8mm, % Bindekorrektur
	headsepline, % Linie unterhalb von Kopfzeile
	footsepline, % Linie überhalb der Fußzeile
%	headinclude, % Kofpzeile und...
%	footinclude, % Fußzeile bei Berechnung von Satzspiegel berücksichtigt (führt zu größerem Rand)
	parskip=full, % Absätze sind deutlicher
%	numbers=noendperiod, % kein automatischer Punkt nach Gliederungsnummer
	headings=small,
	toc=chapterentrydotfill, % Punkte bis zur Seitennummer bei Kapiteln
	toc=listofnumbered,
	toc = bibliographynumbered,
	listof=entryprefix, % Präfix für Einträge in Abbildungs- und Tabellenverzeichnis
	listof=nochaptergap % Abstand für Kapiteleinträge in extra Verzeichnissen
	numbers=noendperiod, % kein automatischer Punkt nach Gliederungsnummer
%	appendixprefix = true
]{scrreprt} % Dokumentenart



%%%% PACKAGES %%%%%
\usepackage[T1]{fontenc} % Europäische Zeichensätze und Silbentrennung
\usepackage[utf8]{inputenc} % deutsche Sonderzeichen
\usepackage[ngerman, english]{babel} % dito

% Schriftart
%\usepackage{lmodern} % Schriftart Latin Modern
\usepackage{mathptmx} % Schriftart Helvetica
\usepackage[scaled=0.9]{helvet}
%\usepackage[scaled=0.78]{luximono} % Typewriter-Schriftart
\usepackage{courier} % Typewriter-Schriftart

%\typearea[current]{calc} % berechne nun Satzspiegel aufgrund der gewählten Schriftart, falls DIV=calc gewählt in \documentclass[options]

\usepackage{amsmath}  % MACHT
\usepackage{amsfonts} %		MATHE
\usepackage{amssymb}  %			MÄCHTIGER
\usepackage{stmaryrd} % ...und noch eine Menge nützlicher Symbole
\usepackage{mathtools}

\usepackage{blindtext} % Kann Blindtext einfügen (macht Formatierungen besser sichtbar)
\usepackage[table]{xcolor} % mehr Optionen für Tabellenumgebungen (zB für Tabellen, aber auch für Pakete wie listings oder pgfplotstable)

\usepackage{listings} % Code-Umgebung. Ermöglicht einbinden von Code-Dateien
\AtBeginDocument{\DeclareCaptionSubType{lstlisting}}

\usepackage{graphicx} % Ermöglicht Einbinden von Grafiken
\usepackage{array} % Verbesserte Array und Tabellen Umgebung

%\usepackage[%
%		showframe,
%	]{geometry}
% Seiteneinrichtung, [showframe] zeigt Seitenspiegel. Auskommentieren, falls Seite über options in \documentclass definiert wird.

\usepackage[chapter]{algorithm} % Definiere Umgebung für...
\usepackage{algpseudocode}      % ...Pseudo-Code
\usepackage{siunitx} % SI-Einheiten und Zahlen in einheitlichem System
\usepackage{pdfpages} % Einbinden von pdfs möglich
\usepackage{booktabs} % Optionen für formale Tabellen in wiss. Arbeiten
\usepackage{tabularx} % Ermöglicht individuellere Tabellen (v.a. Spaltengröße)
\usepackage{pgfplots} % Erstellen von Diagrammen und Plots
\usepackage{pgfplotstable} % Einlesen von Daten (csv, txt) für Plots
\usepackage{tocbasic} %mehr Kontrolle über gleitende Umgebungen, besser als float
\usepackage{scrhack} % Kompatibilität für float bzw. tocbasic für ältere Schnittstellen, zB in listings zu finden. Dieses Paket unterbindet eine entsprechende Warnung.

\usepackage[%
babel, % deutsche Sonderzeichen
%german=quotes, % deutsche Anführungszeichen
]{csquotes} % Bilbliographiestil

\usepackage[%
    style=numeric, % Citation style
    doi=true,  
    maxcitenames=2, 
    mincitenames=1, 
    maxbibnames=8, 
    minbibnames=8, 
    uniquelist=false, 
    isbn=false, 
    giveninits=true,
    backend=biber,
    % backref=true, 
    backrefstyle=all+,
]{biblatex}

\usepackage[%
autooneside=false,	% ???
]{scrlayer-scrpage} % ermöglicht individuelles Anpassen von Kopf- und Fußzeile

\usepackage{caption,subcaption} % Mehr Einstellungen für Beschriftungen
\usepackage{tikz} % Zeichenprogramm
\usepackage{eurosym} % Eurozeichen
\usepackage{enumitem} % verbesserte Nummerierungs- und Aufzählungsumgebung
\usepackage{multicol} % ermögliche mehrere Spalten
\usepackage{letltxmacro} % Verbesserte Funktionalität, um interne Macros umdefinieren zu können
\usepackage{xparse} % Mehr Funktionalität für Macros und eigene Kommandos
\usepackage[absolute]{textpos} % Positionierung von Textblöcken unabhängig von Seitenrändern
\usepackage{calc} % Berechnungen
\usepackage{tabto} % Tabulatoren
\usepackage[%
	page,
	toc,
	titletoc,
	title,
]{appendix} % Mehr Befehle für den Anhang
\usepackage{etoolbox}
\usepackage[htt]{hyphenat} % konsequentere Silbentrennung für alle Fonts

\usepackage{hyperref} % Nutz- und Sichtbare Hyperlinks, META-Datendes PDFs
\usepackage[ngerman, english]{cleveref} % Mehr Optionen für Querverweise

\usepackage{microtype} % Kleinere Layout Optimierungen

% Debugging:
%\usepackage{showframe} % Layout-Boxen anzeigen
%\usepackage{layout} % Layout-Informationen
%\usepackage{printlen} % Längenwerte ausgeben
