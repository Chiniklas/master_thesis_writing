%%%%%%%%%%%%%%%%%%%%%%%%%%%%%%%%%%%%%%%%%%%%%%%%%%%%%%%%%%%%%%%%%%%%%%%%%%%%%%%%
% PACKAGE SETUP
%%%%%%%%%%%%%%%%%%%%%%%%%%%%%%%%%%%%%%%%%%%%%%%%%%%%%%%%%%%%%%%%%%%%%%%%%%%%%%%%



%% Erstellt pdf 1.6 statt 1.5 und verhindert damit Warnungen, falls pdfs v1.6 eingebunden werden
\pdfminorversion=6



% Speichert plots und tikz-Diagramme in externe pdf's in den gewählten Unterordner. Bbenötigt den '--enable-write18'-Befehl in pdfLaTeX, so zB kann die Befehlsstruktur aussehen:
% "pdflatex.exe -synctex=1 -interaction=nonstopmode --enable-write18 %.tex"
\usepgfplotslibrary[external, fillbetween, colormaps]
\usetikzlibrary{%
	external,
	shapes.geometric,
	arrows.meta,
	shapes.arrows,
	fit,
	matrix,
	positioning,
	shapes.misc,
	shadows,
}
\tikzexternalize[prefix=04_TikZ_Externalized/]



%% pgfplots: 
\pgfplotsset{%
	compat=1.15, % Kompatibilität
}


%% siunitx
\sisetup{%
	locale=DE, % Deutsche Einheitenschreibweise: m s^-1 statt m/s
	output-decimal-marker = {.} %  Punkt statt Komma als Dezimaltrenner (Geschmackssache)
}


%%	hyperref Einstellungen - Links und PDF-Meta-Daten
% Solange die Arbeit noch nicht fertig ist am Besten auskommentieren, da in den Voreinstellungen die Verweise am deutlichsten sind.
\hypersetup{%
	pdfborder=0 0 .6, % Rahmendicke der Links. Adobe stellt <.6 nicht ordnungsgemäß dar
	%	colorlinks=true,
	%	citecolor=black,
	%	urlcolor=black,
	%	linkcolor=black,
	%	pdffitwindow=true,
	%	plainpages=false,
}


%% Anpassung der Literaturverzeichnisse
\DefineBibliographyStrings{ngerman}{%
	bibliography = {Literaturverzeichnis}, % Name
	techreport = {Datenblatt},
	mathesis = {Masterarbeit},
	phdthesis = {Dissertation},
	andothers = {{et\,al\adddot}}, % schreibe et al. statt u.a.
}


%% listings	
\lstset{%
%	language=Matlab, % Standardsprache zur Interpretation
	breaklines=true, % Zeilenumbruch automatisch einfügen
	tabsize=4, % Anzahl Leerzeichen pro Tabulator
	inputencoding=latin1, % bindet Umlaute im Code korrekt ein
	basicstyle=\ttfamily, % nutzt Typewriter-Schriftart (falls def. zB. Luximono)
	fontadjust=true,
	columns=flexible, % Anpassung der Zeichenabstände auf Zeilenbreite
	numbers=left, % Zeilennummern links
	numberstyle=\tiny, % Schriftgröße Zeilennummern
	stepnumber=1, % Schrittgröße der dargestellten Zeilennummern
	numbersep=5pt, % Abstand der Zeilennummern vom Code
	firstnumber=1,
	%	firstnumber=last, % fortlaufende Nummerierung
	backgroundcolor=\color[gray]{.98}, % Hintegrund Rahmen
	frame=single,  % Rahmenart
	framerule=0.1pt, % Rahmendicke
	showstringspaces=false, % Hervorhebung von Leerzeichen in strings abschalten
	commentstyle=\color{gray},%\textit, % definiere Aussehen von Kommentaren
	stringstyle=\color{black!20!green}, % Definiere Aussehen von Strings
	morekeywords={rng, realsqrt, ones, zeros}, % Definiere weitere Schlüsselwüörter zur Hervorhebung im Code
}
%\renewcommand{\lstlistingname}{Algorithmus} % Listing heißt nun Algorithmus (German only)





%%%%%%%%%%%%%%%%%%%%%%%%%%%%%%%%%%%%%%%%%%%%%%%%%%%%%%%%%%%%%%%%%%%%%%%%%%%%%%%%
% DOCUMENT SETUP
%%%%%%%%%%%%%%%%%%%%%%%%%%%%%%%%%%%%%%%%%%%%%%%%%%%%%%%%%%%%%%%%%%%%%%%%%%%%%%%%


% Verhindert Schusterjungen und Hurenkinder
\clubpenalty10000
\widowpenalty10000
\displaywidowpenalty10000


%% Erstellt Kapitelnamen in Kopf bzw. Seitenzahlen in Fußzeile
\pagestyle{scrheadings}
\ofoot[\pagemark]{\pagemark} % Erstellt Seitenzahlen jeweils außen
\renewcommand*{\headfont}{\normalfont} % Kopfzeilen sind nicht kursiv dargestellt.
\renewcommand{\chapterpagestyle}{scrheadings}


% Inhaltsverzeichnis
\makeatletter
\renewcommand{\@dotsep}{.3} % Abstand der Füllpunkte


% Überschrift chapter
\setkomafont{chapter}{\sffamily\bfseries\fontsize{16pt}{19pt}\selectfont}
\RedeclareSectionCommand[%
beforeskip=15pt,
afterskip=15pt
]{chapter}


% Überschrift section
\setkomafont{section}{\sffamily\bfseries\fontsize{14pt}{16pt}\selectfont}
\RedeclareSectionCommand[%
beforeskip=25pt,
afterskip=15pt
]{section}


% Überschrift subsection
\setkomafont{subsection}{\sffamily\bfseries\fontsize{12pt}{13pt}\selectfont}
\RedeclareSectionCommand[%
beforeskip=25pt,
afterskip=5pt
]{subsection}


% Tabellen Beschriftung
\captionsetup[table]{%
	format=hang,
	labelfont = it,
	labelsep=colon,
	textfont = it,
	singlelinecheck=off,
	skip=3pt,
}


% Abbildungsbeschriftung
\captionsetup[figure]{%
	format=hang,
	labelfont = it,
	labelsep=colon,
	textfont = it,
	singlelinecheck=off,
	skip=6.6mm,
}


% Algorithmusbeschriftung (Quellcode)
\captionsetup[lstlisting]{%
	format=hang,
	labelfont = it,
	labelsep=colon,
	textfont = it,
	singlelinecheck=off,
	skip=3pt,
}


% Algorithmusbeschriftung (Pseudo-Code)
% remove upper rule
\makeatletter
\newcommand\fs@ruled@notop{\def\@fs@cfont{\bfseries}\let\@fs@capt\floatc@ruled
	\def\@fs@pre{}%
	\def\@fs@post{\kern2pt\hrule\relax}%
	\def\@fs@mid{\kern2pt\hrule\kern2pt}%
	\let\@fs@iftopcapt\iftrue}
\renewcommand\fst@algorithm{\fs@ruled@notop}
\makeatother

% Rename caption for ngerman
%\floatname{algorithm}{Algorithmus}

\captionsetup[algorithm]{%
	format=hang,
	labelfont = it,
	labelsep=colon,
	textfont = it,
	singlelinecheck=off,
	skip=3pt,
}

% Literatur
\setlength\bibitemsep{\itemsep}


% Anhang formatieren
\renewcommand{\autodot}{}%
\newcommand{\appendixTocString}{\appendixname\space\thechapter\autodot}%
\newlength{\appendixTocStringLength}%
\settowidth{\appendixTocStringLength}{\appendixTocString}%
\addtolength{\appendixTocStringLength}{1.5em}%
\makeatletter%
\gappto{\appendix}{%Doing everything in the appendix%
	\patchcmd{\@@makechapterhead}{\endgraf\nobreak\vskip.5\baselineskip}{}{}{}%
	\renewcommand{\autodot}{:}%
	\addtocontents{toc}{%
		\protect\patchcmd{\protect\l@chapter}%
		{1.5em}%
		{\protect\appendixTocStringLength}%
		{}{}}%
	\patchcmd{\@chapter}{\addchaptertocentry{\thechapter}{\scr@ds@tocentry}%
	}{%
		\addchaptertocentry{\appendixTocString}{\scr@ds@tocentry}}{}{}%
}%
\makeatother%

\renewcommand{\appendixpagename}{\appendixname}
\renewcommand{\appendixtocname}{\appendixname}